In November 2020, SWVL launched a long distance ride sharing services by partnering with Matatu operators. The service was launched  in 12 routes, connecting Naivasha, Nakuru, Molo, Eldoret, Narok, Bomet, Kericho, Kisii, Kisumu, Nyeri, Nanyuki and Machakos, \citep{alvin2020}. Swvl was targeting to make the fare prices constant and have timely rides.

Although the service was launched during the nation locked down, Kenyans were eager to try the service. Especially those who have enjoyed their short distance ride sharing services. But in the long distance ride sharing business did not catch up. The Matatu operators would switch from the SWVL service when there was high demand. They needed the flexibility to decided and set the fare prices on their own.

From the experience of SWVL, we can lessons on how the market operates and what needs should be addressed. The Kenyan market is not ready for another `uber' like product for long distance ride sharing, but it needs a ridesharing solution that equips both the driver and the passengers.