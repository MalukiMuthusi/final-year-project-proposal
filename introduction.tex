Ridesharing is also known as liftsharing or car sharing in the UK@. This is different from the terms `carsharing' in North America or `car clubs' in the UK, which refer to short term auto use of a car from a fleet of cars, that are hourly shared by passengers,~\cite{Shaheen2009}.

Ridesharing is the sharing of a cars journey so that one person drives, preventing the need for the other people to drive themselves to the location. The driver and the passenger are travelling towards the same direction.~\cite{Chan2012}. When payment is involved it is not for profitable reasons but to enable to cover the cost and services for the journey.

There has been a lot of interest in the ridesharing services in the recent years. This is because of the use of technology and easy access to internet services. Many people would prefer to travel on a private car than the public vehicles.

Ridesharing is seen as a solution to reducing congestion,  offering quality services to people, and reducing energy consumption~\cite{Noland2006}. Governments have put in place policies to encourage ridesharing services.